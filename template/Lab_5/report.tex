\documentclass{article}
\usepackage{amsmath, amssymb}

\begin{document}

\title{ЛАБОРАТОРНАЯ РАБОТА №5}
\author{}
\date{}
\maketitle

\section*{Вероятностные алгоритмы проверки чисел на простоту}

Пусть \( a \) – целое число. Числа \( \pm1, \pm a \) называются тривиальными делителями числа \( a \).

Целое число \( p \in Z / \{ 0 \} \) называется простым, если оно не является делителем единицы и не имеет других делителей, кроме тривиальных. В противном случае число \( p \in Z / \{-1, 0, 1\} \) называется составным.

Например, числа \( \pm2, \pm3, \pm5, \pm7, \pm11, \pm13, \pm17, \pm19, \pm23, \pm29 \) являются простыми.

Пусть \( m \in N, m > 1 \). Целые числа \( a \) и \( b \) называются сравнимыми по модулю \( m \) (обозначается \( a \equiv b \pmod{m} \)) если разность \( a - b \) делится на \( m \). Также эта процедура называется нахождением остатка от целочисленного деления \( a \) на \( b \).

Проверка чисел на простоту является составной частью алгоритмов генерации простых чисел, применяемых в криптографии с открытым ключом. Алгоритмы проверки на простоту можно разделить на вероятностные и детерминированные.

Детерминированный алгоритм всегда действует по одной и той же схеме и гарантированно решает поставленную задачу (или не дает никакого ответа). Вероятностный алгоритм использует генератор случайных чисел и дает не гарантированно точный ответ. Вероятностные алгоритмы в общем случае не менее эффективны, чем детерминированные (если используемый генератор случайных чисел всегда дает набор одних и тех же чисел, зависящих от входных данных, то вероятностный алгоритм становится детерминированным).

Для проверки на простоту числа \( n \) вероятностным алгоритмом выбирают случайное число \( a \), такое что \( 1 < a < n \), и проверяют условия алгоритма. Если число \( n \) не проходит тест по основанию \( a \), то алгоритм выдает результат «Число \( n \) составное», и число \( n \) действительно является составным.

Если же \( n \) проходит тест по основанию \( a \), ничего нельзя сказать о том, действительно ли число \( n \) является простым. Последовательно проведя ряд проверок таким тестом для разных \( a \) и получив для каждого из них ответ «Число \( n \), вероятно, простое», можно утверждать, что число \( n \) является простым с вероятностью, близкой к 1. После \( t \) независимых выполнений теста вероятность того, что составное число \( n \) будет \( t \) раз объявлено простым (вероятность ошибки), не превосходит \( \frac{1}{2^t} \).

\section*{Тест Ферма}

Тест Ферма основан на малой теореме Ферма: для простого числа \( p \) и произвольного числа \( a \), такого что \( 1 \leq a \leq p - 1 \), выполняется сравнение

\[
a^{p-1} \equiv 1 \pmod{p}.
\]

Следовательно, если для нечетного \( n \) существует такое целое \( a \), что \( 1 \leq a < n, \gcd(a, n) = 1 \) и \( a^{n-1} \not\equiv 1 \pmod{n} \), то число \( n \) составное. Отсюда получаем следующий вероятностный алгоритм проверки числа на простоту.

\subsection*{1. Алгоритм, реализующий тест Ферма}

\textbf{Вход:} Нечетное целое число \( n \geq 5 \).  

\textbf{Выход:} «Число \( n \), вероятно, простое» или «Число \( n \) составное».

\begin{enumerate}
    \item Выбрать случайное целое число \( a \), такое что \( 2 \leq a \leq n - 2 \).
    \item Вычислить \( r \gets a^{n-1} \pmod{n} \).
    \item При \( r = 1 \), результат: «Число \( n \), вероятно, простое». В противном случае результат: «Число \( n \) составное».
\end{enumerate}

\section*{Тест Соловея–Штрассена}

Тест Соловея–Штрассена основан на критерии Эйлера: нечетное число \( n \) является простым тогда и только тогда, когда для любого целого числа \( a \), такого что \( 1 \leq a \leq n - 1 \) и взаимно простого с \( n \), выполняется сравнение:

\[
a^{\frac{n-1}{2}} \equiv \left( \frac{a}{n} \right) \pmod{n},
\]

где \( \left( \frac{a}{n} \right) \) – символ Якоби.

Пусть \( t, n \in \mathbb{Z} \), где \( n = p_1 p_2 \ldots p_r \) и числа \( p_i \neq 2 \) простые (не обязательно различные). Символ Якоби \( \left( \frac{m}{n} \right) \) определяется равенством

\[
\left( \frac{m}{n} \right) = \left( \frac{m}{p_1} \right) \left( \frac{m}{p_2} \right) \ldots \left( \frac{m}{p_r} \right).
\]

\subsection*{2. Алгоритм вычисления символа Якоби}

\textbf{Вход:} Нечетное целое число \( n \geq 3 \), целое число \( a \), такое что \( 0 \leq a < n \).

\textbf{Выход:} Символ Якоби \( \left( \frac{a}{n} \right) \).

\begin{enumerate}
    \item Положить \( g \gets 1 \).
    \item Если \( a = 0 \), результат: 0.
    \item Если \( a = 1 \), результат: \( g \).
    \item Представить \( a \) в виде \( a = 2^k a_1 \), где число \( a_1 \) нечетное.
    \item При четном \( k \), положить \( s \gets 1 \); при нечетном \( k \), положить \( s \gets 1 \), если \( n \equiv \pm1 \pmod{8} \); положить \( s \gets -1 \), если \( n \equiv \pm3 \pmod{8} \).
    \item Если \( a_1 = 1 \), результат: \( g \cdot s \).
    \item Если \( n \equiv 3 \pmod{4} \) и \( a_1 \equiv 3 \pmod{4} \), то \( s \gets -s \).
    \item Положить \( a \gets n \pmod{a_1}, n \gets a_1, g \gets g \cdot s \) и вернуться на шаг 2.
\end{enumerate}

\subsection*{3. Алгоритм, реализующий тест Соловея–Штрассена}

\textbf{Вход:} Нечетное целое число \( n \geq 5 \).

\textbf{Выход:} «Число \( n \), вероятно, простое» или «Число \( n \) составное».

\begin{enumerate}
    \item Выбрать случайное целое число \( a \), такое что \( 2 \leq a < n - 2 \).
    \item Вычислить \( r \gets a^{\frac{n-1}{2}} \pmod{n} \).
    \item При \( r \neq 1 \) и \( r \neq n - 1 \), результат: «Число \( n \) составное».
    \item Вычислить символ Якоби \( s \gets \left( \frac{a}{n} \right) \).
    \item При \( r \neq s \pmod{n} \), результат: «Число \( n \) составное». В противном случае результат: «Число \( n \), вероятно, простое».
\end{enumerate}

На сегодняшний день для проверки чисел на простоту чаще всего используется тест Миллера–Рабина, основанный на следующем наблюдении. Пусть число \( n \) нечетное и \( n - 1 = 2^s r \), где \( r \) – нечетное. Если \( n \) простое, то для любого \( a \geq 2 \), взаимно простого с \( n \), выполняется условие \( a^{p-1} \equiv 1 \pmod{p} \).

\subsection*{4. Алгоритм, реализующий тест Миллера–Рабина}

\textbf{Вход:} Нечетное целое число \( n \geq 5 \).

\textbf{Выход:} «Число \( n \), вероятно, простое» или «Число \( n \) составное».

\begin{enumerate}
    \item Представить \( n - 1 \) в виде \( n - 1 = 2^s r \), где число \( r \) нечетное.
    \item Выбрать случайное целое число \( a \), такое что \( 2 \leq a < n - 2 \).
    \item Вычислить \( y \gets a^r \pmod{n} \).
    \item При \( y \neq 1 \) и \( y \neq n - 1 \), выполнить следующие действия:
    \begin{enumerate}
        \item Положить \( j \gets 1 \).
        \item Если \( j \leq s - 1 \) и \( y \neq n - 1 \), то
        \begin{enumerate}
            \item Положить \( y \gets y^2 \pmod{n} \).
            \item Если \( y = 1 \), результат: «Число \( n \) составное».
            \item Положить \( j \gets j + 1 \).
        \end{enumerate}
        \item При \( y \neq n - 1 \), результат: «Число \( n \) составное».
    \end{enumerate}
    \item Результат: «Число \( n \), вероятно, простое».
\end{enumerate}

\end{document}
