\section*{Лабораторная работа №6}
\subsection*{Разложение числа на множители}

Задача разложения на множители — одна из первых задач, использованных для построения криптосистем с открытым ключом.

Задача разложения составного числа на множители формулируется следующим образом: для данного положительного целого числа $n$ найти его каноническое разложение $n = p_1^{a_1}p_2^{a_2} \ldots p_s^{a_s}$, где $p_i$ — попарно различные простые числа, $a_i \geq 1$.

На практике не обязательно находить каноническое разложение числа $n$. Достаточно найти его разложение на два нетривиальных сомножителя: $n = pq, 1 \leq p \leq q < n$. Далее будем понимать задачу разложения именно в этом смысле.

\subsection*{p-метод Полларда}
Пусть $n$ — нечётное составное число, $S = \{0, 1, \ldots, n-1\}$ и $f : S \to S$ — случайное отображение, обладающее сжимающими свойствами, например $f(x) = x^2 + 1 \pmod{n}$. Основная идея метода состоит в следующем.

Выбираем случайный элемент $x_0 \in S$ и строим последовательность $x_0, x_1, x_2, \ldots$ определяемую рекуррентным соотношением:
\[
x_{i+1} = f(x_i),
\]
где $i \geq 0$, до тех пор, пока не найдём такие числа $i, j$, что $i < j$ и $x_i = x_j$.

Поскольку множество $S$ конечно, такие индексы $i, j$ существуют (последовательность «зацикливается»). Последовательность $\{x_i\}$ будет состоять из «хвоста» $x_0, x_1, \ldots, x_{i-1}$ длины $O\left(\sqrt{\frac{\pi n}{8}}\right)$ и цикла $x_i = x_j, x_{i+1}, \ldots, x_{j-1}$ той же длины.

\subsection*{Алгоритм, реализующий p-метод Полларда}
\textbf{Вход.} Число $n$, начальное значение $c$, функция $f$, обладающая сжимающими свойствами. \\
\textbf{Выход.} Нетривиальный делитель числа $n$.

\begin{enumerate}
    \item Положить $a \gets c, b \gets c$.
    \item Вычислить $a \gets f(a) \pmod{n}, b \gets f(b) \pmod{n}$.
    \item Найти $d \gets \text{НОД}(a-b, n)$.
    \item Если $1 < d < n$, то положить $p \gets d$ и результат: $p$. \\
    При $d = n$ результат: «Делитель не найден»; при $d = 1$ вернуться на шаг 2.
\end{enumerate}

\textbf{Пример.} Найти p-методом Полларда нетривиальный делитель числа $n = 1359331$. Положим $c = 1$ и $f(x) = x^2 + 5 \pmod{n}$. Работа алгоритма иллюстрируется следующей таблицей:

\[
\begin{array}{|c|c|c|c|}
\hline
i & a & b & d = \text{НОД}(a-b, n) \\
\hline
1 & 1 & 1 & 1 \\
2 & 6 & 41 & 1 \\
3 & 41 & 123939 & 1 \\
4 & 1686 & 391594 & 1 \\
5 & 123939 & 438157 & 1 \\
6 & 435426 & 582738 & 1 \\
7 & 391594 & 1144026 & 1 \\
8 & 1090062 & 885749 & 1181 \\
\hline
\end{array}
\]

Таким образом, $1181$ является нетривиальным делителем числа $1359331$.

\subsection*{Метод квадратов}
\textbf{Теорема Ферма о разложении.} Для любого положительного нечётного числа $n$ существует взаимно однозначное соответствие между множеством делителей числа $n$, не меньших, чем $\sqrt{n}$, и множеством пар $\{s, t\}$ таких неотрицательных целых чисел, что $n = s^2 - t^2$.

\textbf{Пример.} У числа $15$ два делителя, не меньших, чем $\sqrt{15}$, это числа $5$ и $15$. Тогда получаем два представления:
\begin{itemize}
    \item $15 = pq = 3 \cdot 5$, откуда $s = 4, t = 1$ и $15 = 4^2 - 1^2$;
    \item $15 = pq = 1 \cdot 15$, откуда $s = 8, t = 7$ и $15 = 8^2 - 7^2$.
\end{itemize}
